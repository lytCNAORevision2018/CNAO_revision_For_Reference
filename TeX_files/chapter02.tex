\chapter{读者需要具备的预备知识清单}
\section{数学知识}
\noindent 除高考数学考纲(以教育部官网2019公布版为准)外,以下提及的知识点,若诸参考资料有差异,以同济大学第七版《高等数学》、第六版《工程数学线性代数》和浙江大学出版社第五版《概率论与数理统计》为准。在阅读本资料并利用其辅助求解问题的过程中应将以下数学知识视作工具,并不一定要求思维逻辑量较大的推导论证。建议有能力的读者在阅读本资料前通过自学或听讲对考研数学(一)范围内的知识点做一个初步的了解,并在实践中不断加深印象、提高使用数学工具的能力。参考书目已经给出。对于使用本资料的读者而言,最需要的能力根据考研数学(一)的模块划分如下。
\paragraph{高等数学部分}
计算一元函数、二元函数的微积分,并讨论其性态;微分方程的求解。
\paragraph{线性代数部分}
将矩阵进行相似对角化、将二次型化成标准型,用矩阵的观点解线性方程组并给出其解的结构。
\paragraph{概率论与数理统计部分}
常见的概率分布、回归分析与方差分析。
\section{物理知识}
\noindent 除高考物理考纲(以教育部官网2019公布版为准)外:

角动量守恒定律、麦克斯韦速度分布律、和理想气体相关、康普顿散射与逆康普顿散射。

注意:对本资料核心内容起到最关键的支撑作用的内容是高考范围内的圆周运动规律及高考视角下的动能定理、功能关系等物理学基本理论。
\section{其他知识}
\noindent 地理知识:区时的概念和换算。

\noindent 关于我国古代天文、历法的文化常识。参考书目:王力《古代汉语》。需要具备一定的文言文阅读能力。
\section{常识}
\begin{Large}
	常识的范围是无穷大的!
\end{Large}
\chapter{行星与航天器的运动}
\section{行星和太阳的视运动}
\subsection{行星的视运动}
\paragraph{行星相对于太阳的视运动}
以下的地内行星特指水星、金星。在某些情况下可以用此规律,运用类比的逻辑得出符合某些条件的小行星或太阳系小天体的运动规律。
\subparagraph{地内行星相对于太阳的视运动}地内行星在轨道上有四个特殊位置:上合、下和、东大距、西大距。上合和下合,太阳和行星在同一方位,黄经相等;上合时行星到地球比太阳到地球更远,下合时反之。东西大距是行星与太阳角距离最大的位置;地内行星在太阳东边为东大距,在太阳西边则为西大距。上合时,行星和太阳同时升起、同时落下,不可能被看见。之后逐渐向东运动,离开太阳一定角度时,成为昏星。至东大距时,和太阳角距离最大,可见时间最长,高度也最高。东大距后,行星转而向太阳的方向靠拢,和太阳的角距离逐渐减小。到下合,和太阳再次同升同落。再出现时,行星运动到太阳西边,成为晨星。西大距时和太阳的角距离最大,可见时间最长,高度也最高。

水星或金星只有在下合期间可能发生凌日。水星凌日平均每百年发生13次。金星凌日每两次分为一组,每组之间的两次相隔八年,但是组与组之间间隔超过一个世纪。
\subparagraph{地外行星相对于的视运动}地外行星也有四个特殊位置:合、冲、东方照、西方照。合日时,行星和地球距离最远,行星的黄经和太阳相同,二者同升同落,不可见;东方照时,行星在太阳东边,黄经比太阳黄经小$90^\circ$,半夜升出地平,后半夜课间;冲时,行星的黄经和太阳相差$180^\circ$,整夜可见;西方照时,行星在太阳西边,黄经比太阳的黄经大$90^\circ$,半夜没入地平,前半夜可以观测。

每一次冲,行星和地球的距离不同。距离最近的冲,称作大冲。
\paragraph{行星相对于恒星背景的视运动}由于行星和地球都在运动,相对于恒星背景,行星在天球上的运动,大部分时间沿着黄经增加的方向(和太阳视运动方向相同,即自西向东运动),称作顺行。但有时也反方向运动,称作逆行。顺行、逆行转换期间,行星暂时没有相对于恒星背景的运动,称作留。当行星相对于地球的运动方向和二者连线方向一致,或者说,当地球到行星的视线方向只有平行移动时,行星出现留。地内行星的留发生在大距和下合之间,地外行星的留发生在冲和方照期间。地外行星于合日前后在天球上的运动,相对于太阳是自东向西运动,但由于这种运动的速度远小于太阳相对于恒星自西向东的运动速度,所以行星相对于恒星而言,仍为自西向东运动,是顺行而不是逆行。

地外行星的逆行阶段:留-冲-留。余下部分为顺行。

地内行星的逆行阶段:留-下合-留。余下部分为顺行。

只有五颗黄道附近的亮星可能发生掩星:毕宿五、轩辕十四、角宿一、心宿二、北河三。当月球或者其它天体的黄经和以上五颗亮恒星时,称作合。
\subsection{天体的周日视运动}人站在地球上无法感觉到地球的自转,只能看到天球在反方向旋转,每24小时旋转一圈,此过程成为天体的周日视运动。所有的天体都参与这一运动,沿着和赤道平行的小圆绕天极旋转,只有赤道上的天体才沿着赤道做大圆运动,南北两极不参加周日视运动。

以下以北半球为例做简单介绍。在北半球不同纬度的地方,北天极的高度因纬度的不同而不同。在北极,赤道和地平圈重合,所有的天体都沿着水平的小圆运动,不升也不落;夏天,太阳在地平线以上做水平运动,没有黑夜;冬天,太阳在地平线下水平运动,没有白昼。在赤道,北天极降到地平线处,赤道通过天顶,所有的天体都直上直下的运动,在地平线上下的时间各为12小时,一年四季都昼夜平分。

在某一纬度为$\phi$的地方,北天极的高度等于其地理纬度,所有的天体都倾斜地绕着北极星运动。只有刚好在赤道上的天体才是准确的东升西落。在赤道以北的天体从东偏北处升起,从西偏北处落下,在地平线以上的时间大于在地平线以下的时间;在赤道以南的天体从东偏南处升起,从西偏南处落下,在地平线上的时间小于在地平线下的时间。太阳夏天在赤道以北,日出在东北方向,日落在西北方向,昼长夜短;冬天,太阳在赤道以南,日出在东南方向,日落在西南方向,昼短夜长。
\subsection{太阳的周年视运动}
\paragraph{四季星空的变化}正是因为有太阳的周年视运动,才有四季星空的变化。
\paragraph{太阳周年视运动的黄经变化}太阳周年视运动的黄经变化为:从春分点为零点起算,每个节气的黄经增加$15^\circ$。但是由于地球公转的轨道是椭圆,轨道速度并不均匀,诸节气之间的时间间隔并不均匀。
\paragraph{不同纬度处太阳视运动的轨迹}在两极,所有天体,包括太阳都做水平运动,不升不落。以北极点为例,夏至时,太阳在最高处沿高度为$23.5^\circ$的小圆做水平运动。随着季节变化,此小圆缓慢下落,秋分时和地平线重合,冬至时落至地平线下的最低点,然后再缓慢上升,到春分时和地平线再次重合。

在赤道,太阳总是直上直下地运动。夏至时,在东偏北$\varepsilon$升起,西偏北$\varepsilon$处落下;冬至时在东偏南$\varepsilon$处升起,西偏南$\varepsilon$处落下。春分日和秋分日正午,太阳到达天顶,一年四季中正午太阳高度最高,日出正东,日落正西。一年四季昼夜平分。

\paragraph{日地距离和四季冷暖变化的原因}太阳到地球的距离在一年当中确实是变化的。地球在每年1月4日前后过椭圆轨道上的近日点,公转速度较快;7月4日前后过椭圆轨道上的远日点,公转速度较慢。但是一年中季节冷暖并不是距离差别造成的,而是阳光斜照和直射使然。
\section{开普勒行星运动定律}
\paragraph{万有引力定律}
任意两个质点有通过连心线方向上的力相互吸引。该引力大小与它们质量的乘积成正比与它们距离的平方成反比,与两物体的化学组成和其间介质种类无关。写成数学表达式为:
\begin{equation}
	F=\frac{GMm}{r^2}
\end{equation}
\paragraph{开普勒行星运动定律}
\subparagraph{开普勒第一定律}所有行星绕太阳的轨道都是椭圆,太阳在椭圆的一个焦点上。
\subparagraph{开普勒第二定律}开普勒第二定律:中心天体与环绕天体的连线(称矢径)在相等的时间内扫过相等的面积。即:
\begin{equation}
	vr\sin \theta =\frac{GM}{4\pi ^2} = const
\end{equation}
其中,$\theta$为为行星速度与矢径之间的夹角。
\subparagraph{开普勒第三定律}开普勒在《宇宙谐和论》上对开普勒第三定律的原始表述:绕以太阳为焦点的椭圆轨道运行的所有行星,其各自椭圆轨道半长轴的立方与周期的平方之比是一个常量。常见表述:绕同一中心天体的所有行星的轨道的半长轴的三次方$(a^3)$跟它的公转周期的二次方$(T^2)$的比值都相等,即。
\begin{equation}
	\frac{a^3}{T^2}=\frac{GM}{4 \pi ^2}=const
\end{equation}
严格公式:
\begin{equation}
	\frac{T_{1}^2}{T_{2}^2}=\frac{a_{1}^3(M+m_{2})}{a_{2}^3(M+m_{1})}
\end{equation}
\begin{equation}
	\frac{a^3}{T^2(M+m)}=\frac{G}{4 \pi ^2}
\end{equation}
\section{轨道根数与二体问题初步}
根据如下描述建立轨道坐标系:运动限制在轨道平面内,原点为力心所在的焦点F。横坐标轴指向近点A。
\subsection{偏近点角和平近点角}
天体向径:
\begin{equation}
	\vec{\mathbf{r}}=\begin{bmatrix}
		r\cos f\\
		r \sin f
	\end{bmatrix}
\end{equation}
考虑椭圆轨道是从纵轴方向按照$\frac{b}{a}=\sqrt{1-e^2}$进行压缩:
压缩变换将辅助圆上的$C(x,y)$变换到椭圆上的$m(x,y)$。当天体沿着轨道运动时,像点C沿着辅助圆运动。在一个周期的时间内,像点和天体同时从近点出发并返回近点。记像点C关于中心O的向径$\vec{OC}$的幅角为E。
像点的向径:
\begin{equation}
	\vec{\mathbf{r^\prime}}=\begin{pmatrix}
		x^\prime \\
		y^\prime \\
	\end{pmatrix}=a	\begin{pmatrix}
		\cos E -e\\
		\sin E\\	
		\end{pmatrix}
\end{equation}
可知m点的向径为:
\begin{equation}
	\vec{\mathbf{r}}=a\begin{pmatrix}
		\cos E-e\\
		\sqrt{1-e^2}\sin E\\
	\end{pmatrix}
\end{equation}
像点向径OC的幅角称作天体的偏近点角E,力心F和像点C向量的幅角称作天体的平近点角f。
以偏近点角表达中心距和真近点角的公式:
\begin{equation}
	r=a(1-\cos E)
\end{equation}
\begin{equation}
	\begin{pmatrix}
		\cos f\\
		\sin f\\
	\end{pmatrix}
=\frac{1}{1-e\cos E}\begin{pmatrix}
	\cos E-e\\
	\sqrt{1-e^2}\sin E
\end{pmatrix}
\end{equation}
\begin{equation}
	\tan \frac{f}{2}=\sqrt{\frac{1+e}{1-e}}\tan \frac{E}{2}
\end{equation}
使用真近点角表达偏近点角的关系式:
\begin{equation}
	\begin{pmatrix}
		\cos E\\
		\sin E\\
	\end{pmatrix}
=\frac{1}{1+e\cos f}\begin{pmatrix}
	\cos f+e\\
	\sqrt{1-e^2}\sin f
\end{pmatrix}
\end{equation}

速度向量:
\begin{equation}
	\vec{v}=a\frac{{\rm d} E}{{\rm d} t}\begin{pmatrix}
		-\sin E\\
		\sqrt{1-e^2}\cos E\\
	\end{pmatrix}
\end{equation}
对于椭圆轨道:
\begin{equation}
	\frac{{\rm d} M}{{\rm d} t}=n=\frac{2\pi}{T}
\end{equation}
\begin{equation}
	\frac{{\rm d} E}{{\rm d} t}=\frac{na}{r}=\frac{2\pi a}{Tr}
\end{equation}
\begin{equation}
	\frac{{\rm d} f}{{\rm d} t}=\frac{nab}{r^2}=\frac{2\pi ab}{Ta^2}
\end{equation}
\paragraph{轨道的整体计算过程}
轨道运动的整体计算过程如下:

\noindent $\left(1\right)$计算平均运动\begin{equation}
	n=\sqrt{\frac{\mu}{a^3}}
\end{equation}

\noindent $\left(2\right)$计算平近点角$M$。

\noindent $\left(3\right)$通过解开普勒方程求偏近点角$E$。

\noindent $\left(4\right)$计算位置向量$\vec{\mathbf{r}}$

\noindent $\left(5\right)$计算日心距——$|\vec{\mathbf{r}}|$

\noindent $\left(6\right)$计算速度向量。

\paragraph{开普勒方程}
任意时刻$t$,平近点角和偏近点角分别为$M$和$E$,轨道周期为$T$,过近日点时刻为$\tau$:
\begin{equation}
	E -esinE=M
\end{equation}
其中,
\begin{equation}
	M=\frac{2\pi}{T}\left(t-\tau \right)
\end{equation}
\paragraph{欧拉方程}
$\theta$为真近点角:
\begin{equation}
	\theta  = 2 \arctan \left(\sqrt{\frac{1+e}{1-e}}\tan \frac{E}{2}\right)
\end{equation}
\paragraph{Lambert定理}
已知椭圆半长轴$a$和轨道上$P_{1}$和$P_{2}$处的中心距离$r_{1}$和$r_{2}$,
\begin{equation}
	n\Delta t = (E_{2}-E_{1})-e(\sin E_{2}-\sin E_{1})
\end{equation}
引入新变量:
\begin{equation}
	\cos \frac{\alpha_{2}+\alpha_{1}}{2}=e\cos \frac{E_{2}+E_{1}}{2}
\end{equation}

\begin{equation}
	\alpha_{2}-\alpha_{1}=\frac{E_{2}-E_{1}}{2}
\end{equation}
可以写成:

\begin{equation}
	n \Delta t=(\alpha_{2}-\alpha_{1})-(\sin \alpha_{2}-\sin \alpha_{1})
\end{equation}
同时:

\begin{equation}
	r_{1}+r_{2}=2a\left(1-\cos \frac{\alpha_{1}+\alpha_{2}}{2}\cos \frac{\alpha_{1}-\alpha_{2}}{2}\right)
\end{equation}
\begin{equation}
	s=2a\sin \frac{\alpha_{1}+\alpha_{2}}{2}\sin\frac{\alpha_{1}-\alpha_{2}}{2}
\end{equation}
\begin{equation}
	\sin \frac{\alpha_{1}}{2}=\frac{1}{2}\sqrt{\frac{r_{1}+r_{2}+s}{a}}
\end{equation}	

\begin{equation}
	\sin \frac{\alpha_{2}}{2}=\frac{1}{2}\sqrt{\frac{r_{1}+r_{2}-s}{a}}
\end{equation}

关于抛物线的定理:
若$\vec{\mathbf{r_{1}}}$和$\vec{\mathbf{r_{2}}}$为抛物线轨道上的两条向径,$\mathbf{s}=\vec{\mathbf{r_{2}}}-\vec{\mathbf{r_{1}}}$,$r_{1}$,$r_{2}$,$s$分别为该三角形对应的三条边,约定:
$0 \leq f_{3}-f_{1} \leq \pi$,有:
\begin{equation}
	\frac{q}{\sqrt{r_{1}r_{2}}}\left(1+E_{p1}E_{p2}\right)=\cos \frac{f_{2}-f_{1}}{2}
\end{equation}
\begin{equation}
	2q\left(1+E_{p1}E_{p2}\right)=\sqrt{\left(r_{1}+r_{2}+s\right)\left(r_{1}+r_{2}-s\right)}
\end{equation}
\begin{equation}
	E_{p2}-E_{p1}=\frac{1}{\sqrt{2q}}\left(\sqrt{r_{1}+r_{2}+s}-\sqrt{r_{1}+r_{2}-s}\right)
\end{equation}
约定:$0\leq f_{3}-f_{1}\leq \pi $,抛物线运动的Euler方程:
\begin{equation}
	6\sqrt{\mu}(t_{2}-t_{1})=(r_{1}+r_{2}+s)^{\frac{3}{2}}-(r_{1}+r_{2}-s)^{\frac{3}{2}}
\end{equation}
\paragraph{椭圆轨道上的极坐标方程}
\begin{equation}
	r =\frac{a(1-e^2)}{1+e\cos f}
\end{equation}
或者:
\begin{equation}
	r = a(1-\cos E)
\end{equation}
\paragraph{轨道的空间方位}
\subparagraph{平面直角坐标系的旋转}
%在平面上建立坐标系:$\left{O;\vec{e_{1},\vec{e_{2}}}\right}$
%逆时针旋转角度$\psi$,得到新的坐标系:${O;\vec{e_{1}^\prime,\vec{e_{2}^\prime}}}$有:
\begin{equation}
	R(\psi)=\begin{pmatrix}
		\cos \psi&\sin \psi \\
		-\sin \psi &\cos \psi \\
	\end{pmatrix}
\end{equation}
变化式:
\begin{equation}
	(\vec e_{1}^\prime \ \vec e_{2}^\prime)=(\vec e_{1} \ \vec e_{2})R(-\psi)
\end{equation}

\subparagraph{空间直角坐标系的旋转}
考虑三维情形,三维空间的向径可以表示为:
\begin{equation}
	\boldsymbol{r}=(\boldsymbol{e_{1}} \ \boldsymbol{e_{2}} \ \boldsymbol{e_{3}})\begin{pmatrix}
		x\\
		y\\
		z\\
	\end{pmatrix}
\end{equation}
向量$\boldsymbol{e_{1}}$不动,($\boldsymbol{e_{2}}$,$\boldsymbol{e_{3}}$)以$\boldsymbol{e_{1}}$为轴在平面内逆时针旋转角度$\phi$到($\boldsymbol{e_{2}^\prime}$,$\boldsymbol{e_{3}^\prime}$),标架变换可以写成:
\begin{equation}
	(\boldsymbol{e_{1}},\boldsymbol{e_{2}^\prime},\boldsymbol{e_{3}^\prime})=(\boldsymbol{e_{1}},\boldsymbol{e_{2}},\boldsymbol{e_{3}})\boldsymbol{R_{1}}(-\phi)
\end{equation}
变换矩阵写成:
\begin{equation}
	\boldsymbol{R_{1}}(\phi)=\begin{pmatrix}
		1&0&0\\
		0&\cos \phi &\sin \phi \\
		0&-\sin \phi &\cos \phi \\
	\end{pmatrix}
\end{equation}

向量$\boldsymbol{e_{2}}$不动,($\boldsymbol{e_{1}}$,$\boldsymbol{e_{3}}$)以$\boldsymbol{e_{2}}$为轴在平面内逆时针旋转角度$\phi$到($\boldsymbol{e_{1}^\prime}$,$\boldsymbol{e_{3}^\prime}$),标架变换可以写成:
\begin{equation}
	(\boldsymbol{e_{1}^\prime},\boldsymbol{e_{2}},\boldsymbol{e_{3}^\prime})=(\boldsymbol{e_{1}},\boldsymbol{e_{2}},\boldsymbol{e_{3}})\boldsymbol{R_{2}}(-\phi)
\end{equation}
变换矩阵写成:
\begin{equation}
	\boldsymbol{R_{2}}(\phi)=\begin{pmatrix}
		\cos \phi &0&-\sin \phi \\
		0&1&0\\
		\sin \phi &0&\cos \phi \\
	\end{pmatrix}
\end{equation}

向量$\boldsymbol{e_{3}}$不动,($\boldsymbol{e_{1}}$,$\boldsymbol{e_{2}}$)以$\boldsymbol{e_{3}}$为轴在平面内逆时针旋转角度$\phi$到($\boldsymbol{e_{1}^\prime}$,$\boldsymbol{e_{2}^\prime}$),标架变换可以写成:
\begin{equation}
	(\boldsymbol{e_{1}^\prime},\boldsymbol{e_{2}^\prime},\boldsymbol{e_{3}})=(\boldsymbol{e_{1}},\boldsymbol{e_{2}},\boldsymbol{e_{3}})\boldsymbol{R_{3}}(-\phi)
\end{equation}


\begin{equation}
	\boldsymbol{R_{3}}(\phi)=\begin{pmatrix}
		\cos \phi &\sin \phi &0\\
		-\sin \phi &\cos \phi &0\\
		0&0&1\\
	\end{pmatrix}
\end{equation}
旋转角:
\begin{equation}
	(\boldsymbol{e_{1}} \ \boldsymbol{e_{2}} \ \boldsymbol{e_{3}})=(\boldsymbol{e_{1}^\prime} \ \boldsymbol{e_{2}^\prime}  \ \boldsymbol{e_{3}^\prime}){R_{1}}(-\alpha){R_{2}}(-\beta){R_{3}}(-\gamma)
\end{equation}
\begin{equation}
	\begin{pmatrix}
		x^\prime\\
		y^\prime \\
		z^\prime \\
	\end{pmatrix}
=\boldsymbol{R_{3}}(-\Omega)\boldsymbol{R_{2}}(-I)\boldsymbol{R_{1}}(-\omega)
\begin{pmatrix}
	x\\
	y\\
	z\\
\end{pmatrix}	
\end{equation}
上式中,$\Omega$,$I$,$\omega$称作欧拉角。一般旋转矩阵矩阵:
\begin{equation}
	\boldsymbol{R}(\omega,I,\Omega)=\boldsymbol{R_{3}}(\omega)\boldsymbol{R_{1}}(I)\boldsymbol{R_{2}}(\Omega)
\end{equation}
一般旋转矩阵一定可逆,其逆矩阵:
\begin{equation}
	\boldsymbol{R}^{-1}(\omega,I,\Omega)=\boldsymbol{R_{3}}(-\omega)\boldsymbol{R_{1}}(-I)\boldsymbol{R_{2}}(-\Omega)
\end{equation}
\subparagraph{}
$\Omega$:沿着黄道度量,表示黄道和赤道升交点的位置,叫做升交点黄经。

$\omega$:沿着轨道度量,表示升交点到近日点的角距离,叫做近日点幅角。

$i$:表示轨道面对于黄道面倾斜的角度,称作轨道倾角。
从黄道标架$(\hat{\boldsymbol{e_{1}}} \ \hat{\boldsymbol{e_{2}}} \ \hat{\boldsymbol{e_{3}}})$到轨道标架$(\hat{\boldsymbol{e}} \ \hat{\boldsymbol{p}} \ \hat{\boldsymbol{h}})$的变换为:
\begin{equation}
	(\hat{\boldsymbol{e}} \ \hat{\boldsymbol{p}} \ \hat{\boldsymbol{h}})=(\hat{\boldsymbol{e_{1}}} \ \hat{\boldsymbol{e_{2}}} \ \hat{\boldsymbol{e_{3}}})\boldsymbol{R_{3}}(-\omega)\boldsymbol{R_{1}}(-I)\boldsymbol{R_{2}}(-\Omega)
\end{equation}
黄道面标架下,轨道面的法向量:
\begin{equation}
	\hat{\boldsymbol{h}}=\begin{pmatrix}
		\sin i\sin \Omega\\
		-\sin i\cos \Omega \\
		\cos i\\
	\end{pmatrix}
\end{equation}
轨道面标架下,黄道面的法向量:
\begin{equation}
		\hat{\boldsymbol{e_{3}}}=\begin{pmatrix}
		\sin i\sin \omega\\
		\sin i\cos \omega \\
		\cos i\\
	\end{pmatrix}
\end{equation}

\section{和高考有一定关联的另外几个知识点}
\subsection{活力公式}
\paragraph{活力公式}
活力公式是能量守恒定律的直接反映。
航天器的比能量(即单位质量所具有的能量)可以写成:
\begin{equation}
	\varepsilon =\frac{1}{2}v^2-\frac{\mu}{r}
\end{equation}
椭圆运动的总比能量为$-\frac{\varepsilon}{2a}$,双曲线轨道的总比能量为$\frac{\varepsilon}{2a}$。

\noindent 当运动的轨道为椭圆轨道或圆轨道,偏心率范围$\left[0,1\right)$:
\begin{equation}
	v^2=\mu \left(\frac{2}{r}-\frac{1}{a}\right)
\end{equation}
当运动的轨道为双曲线轨道,偏心率范围为$\left[1,+\infty \right)$:
\begin{equation}
	v^2 = \mu \left(\frac{2}{r}+\frac{1}{a}\right)
\end{equation}

其中,$\mu = G(M+m)$。

对于正圆轨道,$a = r$,环绕速度:

\begin{large}
	$v_{0} =\sqrt{\frac{G(M+m)}{r}}$ 
\end{large},
同时可以得到黄金代换公式$GM=gr^2$。
对于抛物线轨道,活力公式中$a = \infty$,$v_{1}=\sqrt{2}v_{0}$。
\paragraph{不同速度下不同类型的轨道判断}
$v < v_{0}$:取远点在该处的椭圆轨道。

$v = v_{0}$:取正圆轨道

$v_{1}>v>v_{0}$:取近点在该处的椭圆轨道。

$v= v_{1}$:取抛物线轨道,来去方向平行。

$v>v_{1}$:双曲线轨道,来去相差一个角度。
\subparagraph{其他结论}
\begin{equation}
	v_{near} =\sqrt{\frac{G(M+m)}{a}\left(\frac{1+e}{1-e}\right)}
\end{equation}

\begin{equation}
	v_{far} = \sqrt{\frac{G(M+m)}{a}\left(\frac{1-e}{1+e}\right)}
\end{equation}


\subsection{行星际飞行器的轨道转移问题初步}
\paragraph{轨道机动}以下只考虑航天器在二体运动轨道上运行的情况。
只有速度的法向分量能够改变轨道平面。
拉普拉斯向量描述了卫星轨道的性状及其在轨道平面上的方位。
\begin{equation}
	\boldsymbol{e}=\frac{1}{\mu}\times \boldsymbol{h}-\boldsymbol{\hat{h}}
\end{equation}
拉普拉斯向量的增量:
\begin{equation}
	\boldsymbol{\Delta e}=\frac{1}{\mu}(\Delta \boldsymbol{v}\times \boldsymbol{h}+\boldsymbol{v}\times \boldsymbol{\Delta h})
\end{equation}
一般的,拱线将在轨道平面内转过一个角度。如果机动发生在近点或远点,且速度只改变大小而不改变方向,有:
$v=v\boldsymbol{\hat{\theta}}$,
$\Delta v=\Delta v\boldsymbol{\hat{\theta}}$,
$\boldsymbol{\hat{h}}=rv\boldsymbol{\hat{h}}$,
$\boldsymbol{\Delta\hat{h}}=r\Delta v\boldsymbol{\hat{h}}$,
因此,轨道偏心率的变化量:
\begin{equation}
	e=\left\{
	\begin{aligned}
		\frac{2}{\mu}rv\Delta v \quad \mbox{在近点}\\
		-\frac{2}{\mu}rv\Delta v \quad \mbox{在远点}\\
	\end{aligned}
	\right
	.
\end{equation}
\paragraph{转移至圆轨道、由地球轨道逃逸}
\subparagraph{转移至圆轨道}
卫星在近地点处的向径:
\begin{equation}
	r_{P}=a(1-e)
\end{equation}
卫星在远地点处的向径:
\begin{equation}
	r_{A}=a(1+e)
\end{equation}
卫星在近地点处的速度满足:
\begin{equation}
	v_{P}^2=\frac{\mu}{a}\frac{1+e}{1-e}
\end{equation}
卫星在远地点处的速度满足:
\begin{equation}
	v_{A}^2=\frac{\mu}{a}\frac{1-e}{1+e}
\end{equation}
先考虑近地点机动,此时转移至半径为$r_{P}$的圆轨道,卫星在此轨道上运行的速度满足:
\begin{equation}
	v_{PC}^2=\frac{\mu}{r_{P}}=\frac{\mu}{a(1-e)}
\end{equation}
半径$r_{A}$的圆轨道速度为:
\begin{equation}
	v_{AC}^2=\frac{\mu}{r_{A}}=\frac{\mu}{a(1+e)}
\end{equation}
在近地点处转移至圆轨道,需要的速度增量为:
\begin{equation}
	\Delta v_{A}=v_{AC}-v_{A}=v_{AC}(1-\sqrt{1-e})
\end{equation}
速度增量和原来椭圆轨道的偏心率大致成正比,偏心率越小,转移至圆轨道所需要的速度增量越小。

\subparagraph{由地球轨道逃逸}
由于抛物线轨道的总能量为零,航天器在近地点和远地点逃逸时的比动能分别为:
\begin{equation}
	\frac{1}{2}v_{PE}^2=\frac{\mu}{a(1-e)}
\end{equation}
\begin{equation}
	\frac{1}{2}v_{AE}^2=\frac{\mu}{a(1+e)}
\end{equation}
所需的速度增量分别为:
\begin{equation}
	\Delta v_{P}=v_{PE}-v_{P}=\sqrt{\frac{\mu}{a}}\left(\frac{\sqrt{2}}{\sqrt{1-e}}-\frac{\sqrt{1+e}}{\sqrt{1-e}}\right)
\end{equation}
\begin{equation}
	\Delta v_{A}=v_{AE}-v_{A}=\sqrt{\frac{\mu}{a}}\left(\frac{\sqrt{2}}{\sqrt{1+e}}-\frac{\sqrt{1-e}}{\sqrt{1+e}}\right)
\end{equation}
\begin{equation}
	\Delta v_{A}-\Delta v_{P}=\sqrt{\frac{2\mu}{a}}(\sqrt{2}-1)e
\end{equation}
之后,容易看出,对于椭圆停泊轨道,不管偏心率是多少,从近地点逃逸所需的速度增量比从远地点逃逸所需的小,因此逃离地球的最佳地点是轨道上最靠近地球的地方。显然,轨道偏心率越大,从远地点和近地点逃逸所需的速度增量之差越大。
\subparagraph{近圆轨道之间的转移:霍曼转移轨道}
转移轨道的总比能量由下式确定:
\begin{equation}
	\varepsilon=-\frac{\mu}{2a}
\end{equation}
原有的轨道半径为$R_{1}$,转移后的轨道半径为$R_{2}$,霍曼转移轨道的半长径和偏心率可以表示为:
\begin{equation}
	R_{1}=a-ae
\end{equation}
\begin{equation}
	R_{2}=a+ae
\end{equation}
由此,$1-e=\frac{R_1}{a}$,$1+e=\frac{R_{2}}{a}$
由初始的圆轨道转移至霍曼转移轨道需要增加的速度:
\begin{equation}
	v_{1}=\sqrt{\frac{\mu}{R_{1}}}
\end{equation}
椭圆轨道近点处的速度:
\begin{equation}
	v_{P}^2=\frac{\mu}{a}\left(\frac{1+e}{1-e}\right)=\frac{2\mu R_{2}}{R_{1}(R_{1}+R_{2})}
\end{equation}
由此,
\begin{equation}
	\Delta v_{1}=v_{1}\left(\sqrt{\frac{2R_{2}}{R_{1}+R_{2}}}-1\right)
\end{equation}
到达霍曼转移轨道和待进入的圆轨道的交点时,还需要另外一个速度增量:
\begin{equation}
	\Delta v_{2}=v_{2}-v_{A}
\end{equation}
椭圆轨道远点处的速度:
\begin{equation}
	v_{P}^2=\frac{\mu}{a}\left(\frac{1-e}{1+e}\right)=\frac{2\mu R_{1}}{R_{2}(R_{1}+R_{2})}
\end{equation}
所以,
\begin{equation}
	\Delta v_{2}=v_{1}\sqrt{\frac{R_{1}}{R_{2}}}\left(1-\sqrt{\frac{2R_{1}}{R_{1}+R_{2}}}\right)
\end{equation}
在两条圆轨道之间转移需要的总的速度增量:
\begin{equation}
	\Delta v =\Delta v_{1}+\Delta v_{2}
\end{equation}
转移轨道的周期由开普勒第三定律给出:
\begin{equation}
	T=\pi\sqrt{\frac{(R_{1}+R_{2})^3}{2\mu}}
\end{equation}
航天器在转移轨道上运行的时间:
\begin{equation}
	\tau =\frac{T}{2}=\frac{\pi}{2v_{0}}\sqrt{\frac{(R_{1}+R_{2})^3}{2\mu}}
\end{equation}
\subsection{拉格朗日点}
以下涉及拉格朗日点位置计算的内容按照高考考纲范围内的要求进行计算论证。在二体问题(限制性平面三体问题)视角下的详细论证请参考《天体测量与天体力学基础》。
各拉格朗日点的位置:
L1:两个大天体的连线之间。

L2:两个大天体之间的连线上,位于较小的天体一侧。

L3:两个大天体之间的连线上,位于较大的天体一侧。

L4:在以两个天体连线为底的等边三角形的第三个顶点上,且在较小天体围绕两天体系统质心运行轨道的前方。

L5:在以两个天体连线为底的等边三角形的第三个顶点上,且在较小天体围绕两天体系统质心运行轨道的后方。


选择日地系统的质心为惯性参考系,太阳的质量,地球的质量,探测器的质量分别为$M,m,\mu$。$x_{n}$为各拉格朗日点和地球轨道的偏离量(其中,拉格朗日点编号和x的下角标一一对应)。假设日地距离为$R$。
日地系统的质心处,满足:
\begin{equation}
	\alpha =\frac{m}{M+m}
\end{equation}
对太阳:
\begin{equation}
	\frac{GMm}{R^2}=M\omega^2\alpha R
\end{equation}
对地球:
\begin{equation}
	\frac{GMm}{R^2}=M\omega ^2R(1-\alpha)
\end{equation}
联立可得角速度:
\begin{equation}
	\omega =\sqrt{\frac{G(M+m)}{R^3}}
\end{equation}
\paragraph{L1点的计算}
\begin{equation}
	G\frac{M\mu}{(R-x_{1})^3}-G\frac{m\mu}{x_{1}^2}=\mu\omega^2(R^2-x_{1}-\alpha R)
\end{equation}
利用恰当的近似公式化简,得:
\begin{equation}
	x_{1}=\sqrt[3]{\frac{m}{3M+m}}R
\end{equation}
\paragraph{L2点的计算}
\begin{equation}
	G\frac{M\mu}{(R+x_{2})^3}+G\frac{m\mu}{x_{2}^2}=\mu\omega^2(R^2+x_{2}-\alpha R)
\end{equation}
利用恰当的近似公式化简,得:
\begin{equation}
	x_{2}=\sqrt[3]{\frac{m}{3M+m}}R
\end{equation}
\paragraph{L3点的计算}
\begin{equation}
	G\frac{M\mu}{(R+x_{3})^3}+G\frac{m\mu}{x_{3}^2}=\mu\omega^2(R^2+x_{3}+\alpha R)
\end{equation}
利用恰当的近似公式化简,得:
\begin{equation}
	x_{3}=-\frac{7m}{12M+5m}R
\end{equation}
\paragraph{L4、L5点的计算}
此处定义:$\theta$为太阳-拉格朗日L4点连线和日地连线之间的夹角。根据对称性,拉格朗日L5点类似。
\begin{equation}
	\left[G\frac{m\mu}{(2R\sin \frac{\theta}{2})^2}\right]^2+\left[\frac{GM\mu}{R^2}\right]^2+\left[2G\frac{m\mu}{(2R\sin\frac{\theta}{2})}\frac{GM\mu}{R^2}\sin \frac{\theta}{2}\right]^2=\mu^2\omega^4\left[R^2+(\alpha R)^2-2\alpha R^2 \cos \theta\right]^2
\end{equation}
化简,得:
\begin{equation}
	(M+m)(2\sin \frac{\theta}{2})^6=M(2\sin\frac{\theta}{2})^3+m
\end{equation}
解这个方程,$\sin \frac{\theta}{2}=\frac{1}{2}$。$\theta =60^\circ$
\subsection{洛希极限与希尔球}
\paragraph{希尔球}
希尔球是环绕在天体周围的、该天体引力占据主导地位的空间。假设大天体和小天体的角标(同时单独为其质量)为$M$和$m$,小天体绕大天体公转的轨道半长轴为$a$,轨道偏心率为$e$,则希尔球半径的近似值:
\begin{equation}
	r=a(1-e)\sqrt[3]{\frac{m}{3M}}
\end{equation}
当小天体绕大天体公转的轨道偏心率可以忽略时,公式可以改写为:
\begin{equation}
	\frac{3r^3}{a^3}=\frac{m}{M}
\end{equation}
可以近似认为,希尔球的直径等于拉格朗日点L1和L2点之间的距离。

在太阳系,海王星有着最大的希尔球,半径为0.775天文单位。

从平面限制型三体问题的角度考虑,请参考《天体测量与天体力学基础》。
\paragraph{洛希极限}
洛希极限是指当行星与卫星距离靠近到一定程度时,潮汐作用会使卫星本身解体分散。该极限最早有法国天文学家洛希计算得出,故得此名。

令洛希极限为$d$,大天体和小天体的角标(同时单独为其质量)为$M$和$m$。对于一个完全刚体、球形的卫星,并忽略其他因素,有:
\begin{equation}
	d=R\sqrt[3]{\frac{2\rho_{M}}{\rho_{m}}}
\end{equation}
当行星达到洛希极限时:
\begin{equation}
	R_{RL}=\sqrt[3]{\frac{9M}{4\pi\rho_{m}}}
\end{equation}
行星的最大希尔球达到该行星和恒星系统内恒星的拉格朗日L1点与L2点,
\begin{equation}
	R_{HS}=R_{RL}\sqrt[3]{\frac{m}{3M}}
\end{equation}
可以写成:
\begin{equation}
	R_{HS}=R_{secondary}=\sqrt[3]{\frac{3m}{4\pi\rho_{m}}}
\end{equation}
此时,行星表面和洛希瓣合一,或说行星充满了希尔球,此时行星不能再吸积物质。

同时应该有:
\begin{equation}
	R_{RL}=R_{secondary}\sqrt[3]{\frac{3M}{m}}=R_{HS}\sqrt[3]{\frac{3M}{m}}
\end{equation}
太阳的平均密度为1.408$g/cm^3$,密度大于$4.224g/cm^3$并围绕太阳公转的天体在太阳系系统内不存在解体风险。
\section{会合周期与行星表面昼夜长短的计算}
\paragraph{会合周期}
地内行星和地外行星的会合周期表达式可以方便地从几何图形中看出。

地内行星:
\begin{equation}
	\frac{1}{S}=\frac{1}{T}-\frac{1}{E}
\end{equation}
地外行星:
\begin{equation}
	\frac{1}{S}=\frac{1}{E}-\frac{1}{T}
\end{equation}
以上二式中,$E=365.2596$天,为地球过近点年的周期。$S$为行星会合周期。

相邻两次之间冲日的时间间隔:
\begin{equation}
	\frac{1}{t}=\frac{1}{T_{planet}}-\frac{1}{T}
\end{equation}
\paragraph{行星表面昼夜长短的计算}
$T_{1}$和$T_{2}$分别为行星的自转和公转周期,则行星的昼夜周期:
\begin{equation}
	T_{0} = \frac{T_{1}T_{2}}{T_{2}-T_{1}}
\end{equation}
某黄赤交角为$\varepsilon$的行星在一个公转周期$T_{g}$中,太阳永不升、落的时间。在行星表面纬度为$\phi$处,当太阳赤纬$\delta$满足$\delta = 90^\circ -\phi$时,
黄经满足:
\begin{equation}
	\lambda =\arcsin \left[\frac{\sin(90^\circ-\phi)}{\sin \varepsilon}\right]
\end{equation}
当太阳的黄经在$\lambda \sim 180^\circ -\lambda$之间,太阳永不落下。这一段时间应当写成:

\begin{equation}
	\Delta T = \frac{T_{g}(180^\circ-2\lambda )}{360^\circ}=\frac{T_{g}\left\{180^\circ-2\arcsin \left[\frac{\sin (90^\circ -\varepsilon)}{\sin \varepsilon}\right]\right\}}{360^\circ}
\end{equation}

\section{潮汐加速度}
\paragraph{潮汐加速度}
月球(或太阳)的引力对于地球表面产生的潮汐加速度(取地心参考系。):
假设$\Delta m$为被吸引的海水质量,地球半径为$R$,被吸引的海水视作质点到月球质心的距离为$d_{0}$,地月质心距离为$d$。则面向月球(太阳)处的海水具有的引力加速度:
\begin{equation}
	a_{front}=\frac{G\Delta m}{(d_{0}-R)^2}-\frac{G\Delta m}{d^2}\approx \frac{2G\Delta mR}{d_{0}^3}
\end{equation}
背向月球(太阳)处所在的方向:
\begin{equation}
	a_{back}=\frac{G\Delta m}{(d_{0}+R)^2}-\frac{G\Delta m}{d^2}\approx -\frac{2G\Delta mR}{d_{0}^3}
\end{equation}

\section{人造卫星的星下点问题}
$t$时刻,卫星星下点的地心纬度为$\psi$,地心经度为$\lambda$,$\theta$是卫星和升交点的角距离,$i$是卫星的轨道倾角,$\omega$为地球自转的角速度,$k$的正负分别对应顺行和逆行轨道。表达式中所有角度,一律使用角度值。
\begin{equation}
	\phi =\arcsin(\sin i \sin \theta)
\end{equation}
\begin{equation}
	\lambda =\lambda_{0}+\arctan (\cos i\tan \theta)-\omega \pm k
\end{equation}
其中,
\begin{equation}
	k = \left\{
	\begin{aligned}
		-180^\circ \quad -180^\circ \leq \theta < -90^\circ\\
		0^\circ \quad -90^\circ \leq \theta \leq 90^\circ\\
		180^\circ \quad 90^\circ < \theta \leq 180^\circ
	\end{aligned}
	\right
	.
\end{equation}

\section{行星的温度与视星等}
\paragraph{行星的温度}
\begin{equation}
	T_{planet} \sim \sqrt[4]{\frac{L}{D^2}}
\end{equation}

\begin{equation}
	T_{tobalance}=\sqrt[4]{\frac{L_{sun}(1-\alpha)}{16\pi D^2\sigma e}} \approx T\sqrt[4]{\frac{1-\alpha}{4}}\sqrt{\frac{R}{D}}
\end{equation}

\paragraph{行星的视星等}

假设该行星表面的反照率是$\alpha$,行星半径为$R$,和地球之间的距离为$r$,和太阳之间的距离为$D$。此时,地球和太阳相距为$r_{1}$。同时假设行星与太阳的连线和地球与行星的连线的夹角为$\beta$。
$\psi(\beta)$为行星相位之间的归一化函数。
当且仅当$\beta \in \left[0,90^\circ\right]$时,$\psi(\beta)=\cos \beta $。其余情况下此函数值为零。
\begin{equation}
	F_{planet}=\alpha \psi(\beta)R^2\frac{1}{r^2}\frac{L}{4\pi D^2}
\end{equation}

\begin{equation}
	\cos \beta =\frac{D^2+r^2-1}{2Dr}
\end{equation}


\chapter{天球坐标系}
\section{球面三角常用基本公式(附常用近似公式)}
\paragraph{球面三角基本公式}
以下假设球面三角形的三个角为$A,B,C$,三个边为$a,b,c$其中相应的小写字母和大写字母配对。

\noindent 正弦定理:
\begin{equation}
	\frac{\sin a}{\sin A}=\frac{\sin b}{\sin B}=\frac{\sin c}{\sin C}
\end{equation}

\noindent 余弦定理
\begin{equation}
	\cos a=\cos b \cos c+\sin b \sin c \cos A
\end{equation}
\begin{equation}
	\cos A =-\cos B \cos C+\sin B \sin C \cos a 
\end{equation}
五元素公式
\begin{equation}
	\sin a \cos B =\cos b\sin c-\sin b \cos c \cos A 
\end{equation}
\begin{equation}
	\sin A \cos b =\cos B \sin C+\sin B \cos C \cos a
\end{equation}
四元素公式
\begin{equation}
	\cot A \sin B =- \cos B \cos C+\sin c \cot a 
\end{equation}



\paragraph{常用的近似公式}以下假定:$x$和$y$的绝对值远小于1。
\begin{equation}
	(1+x)^a=1+ax,\ln(1+x)=x,\lg(1+x)=0.4343x
\end{equation}
\begin{equation}
	(1+x)(1+y)=1+x+y
\end{equation}
\begin{equation}
	e^x=1+x,\sqrt{1-x}=1-\frac{1}{2}x
\end{equation}
\begin{equation}
	\sin x =\tan x=\arcsin x= \arctan x =x,\cos x=1-\frac{1}{2}x^2
\end{equation}
\section{不同天球坐标系之间的比较}
\begin{center}
	\begin{tabularx}{\textwidth}{|c|X|X|X|X|X|}
		%\caption{不同天球坐标系之间的比较}
	\hline
	坐标系 &地平坐标系&黄道坐标系&赤道坐标系&时角坐标系&银道坐标系\\
	\hline 
	基本轴 &铅垂线&垂直于黄道&地球自转轴&  &垂直于银道面\\
	\hline 
	基本点 &天顶 & 北黄极 & 春分点 &   & 银极 \\
	\hline
	原点 & 南点&春分点&春分点&&银道和赤道的升交点\\
	\hline
	基本大圆&地平圈&黄道&天赤道&天赤道&银道面\\
	\hline 
	极&天顶、天底&黄极&天极&天极&银极\\
	\hline
	坐标度量范围&高度:地平圈到天顶,上正下负。方位角:南点起算,东正西负&黄经度:春分点起算,沿着天体运行方向的反方向度量。黄纬:黄道向黄极量度。&赤经:春分点起逆时针方向度量。赤纬:天赤道向天极量度,南负北正。&过观测者子午圈与天赤道交点面向南,沿着赤道圈顺时针方向按小时度量。&银经:从银道和赤道的升交点逆时针度量。银纬:沿着银经圈南北度量,南正北负。\\
	\hline
	\end{tabularx}
\end{center}
\subparagraph{注释}
赤道坐标系中的赤经和时角坐标系中时角的单位用度、分、秒度量,但是相邻两个单位之间为15进制。余下各坐标系的各单位都使用正常的度、分、秒度量,相邻两个单位之间的进制为60。
\section{天球赤道坐标系本身的运动}
\subsection{岁差和地球自转轴进动}
地球自转轴是倾斜的,地球并不是正球体。根据刚体转动的力学原理,在月球或者太阳引力的作用下,自转轴有被扶正的趋势。但转动的地球产生一种抗力,使地球自转轴不会被扶正,还要保持倾角不变,绕和黄道面垂直的轴缓慢旋转,扫过一个圆锥面,周期为26000年。这一运动称作地球自转轴进动。地轴进动时,和黄道面的倾角不变,但赤道面和黄道面的交线要在黄道面上缓慢转动,造成春分点不断向西移动,每年移动约50角秒,造成以节气为准的回归年短语真正的地球公转周期恒星年,约短20分钟。这一现象成为岁差。

"岁差"一词出自《宋史·律历志》:"虞喜云:尧时冬至日短星昴,今二千七百余年,乃东壁中,则之每岁渐差之所至。" 虞喜,两晋时期天文学家,根据当时的天文观测记录和历代记录比较,发现冬至日黄昏出现于正南方的星宿明显发生了变化。唐尧时代,冬至日过黄昏的中天的星宿是昴宿,但他的时代是壁宿。他意识到了这是冬至点不断向西移动的结果,并明确指出,"天自为天,岁自为岁",即和"天"(恒星)为准的"年"和以冬至节气为准的"岁"是有差别的。这就是岁差。公元前2世纪,岁差由古希腊天文学家伊巴谷发现。
\subsection{岁差产生的后果}
\paragraph{天极绕黄极进动}地轴进动导致天极不能固定在某个恒星的位置,而在天球上沿着一个小圆绕黄极缓慢移动。小圆的半径是黄赤交角,周期和地球进动的周期相等。
\paragraph{恒星的赤经赤纬有微小变化}这种变化不是恒星自身的运动,而是由于赤道坐标系的基本大圆赤道和基本点春分点的岁差运动造成的。当代春分点的位置已经从古希腊时代的白羊座移动到了现在双鱼座靠近宝瓶座的地方,原理与之相同。
\subsection{章动}地球自转轴沿着光滑圆锥面进动,仅仅单独考虑月球或者太阳的作用。对地球自转走施加外力的不仅仅有月球和太阳,还有各大行星,由此可知联合作用力的大小和方向都是复杂多变的。加之地球本身不是刚体,内部质量分布不均匀且不恒定,使地轴的空间运动非常复杂。把自转轴沿着一个光滑圆锥面的等速运动部分称作岁差,而把所有其他复杂的摆动称作章动。现代天体测量中采用的一百零六项章动中最主要的章动主项和黄道、白道的交点变化周期(18.6年)有关。我国古代称这一周期为一章,章动因此得名。
\subsection{黄赤交角的变化和地球极移}黄道面的变化使当前黄极正向天极靠近,黄赤交角每世纪减小约$46.8^\prime$,减小的趋势持续15000年左右,然后转为增大。黄道面变化同样影响春分点的位置,方向是东移而不是西移,称作行星岁差,理论值为$0.13^\prime$。

\section{不同天球坐标系之间的相互转化}
\paragraph{地平坐标系和赤道坐标系之间的转化}
设当地的地理纬度为$\psi$,赤经为$\alpha$,赤纬为$\delta$,时角为$t$,天顶距为$z$,地平方位角为$A$。

赤经和时角之间的关系:地方恒星时为$S$,则:
\begin{equation}
	\alpha = S-t
\end{equation}
\subparagraph{由地平坐标系转化为赤道坐标系}
\begin{equation}
	\sin \delta =\sin \psi \cos z-\cos \phi \sin z\cos A
\end{equation}
\begin{equation}
	\cos \delta\sin t =\sin z\sin A
\end{equation}
\begin{equation}
	\cos \delta \cos t=\sin\psi \sin z \cos A+\cos z\cos t
\end{equation}
\subparagraph{由赤道坐标系转化为地平坐标系}
\begin{equation}
	\cos z = \sin \psi \sin \delta +\cos \psi \cos \delta \cos t
\end{equation}
\begin{equation}
	 \sin z\sin A =\cos \delta\sin t
\end{equation}
\begin{equation}
	\sin z \cos A=-\sin \delta \cos \psi+\cos \delta \sin \psi \cos t
\end{equation}
\paragraph{黄道坐标系和赤道坐标系之间的转化}
设天体的黄经$\lambda$,黄纬$\beta$,赤经为$\alpha$,赤纬为$\delta$,黄赤交角$\varepsilon$。
\subparagraph{由黄道坐标系转化为赤道坐标系}
\begin{equation}
	\sin \delta =\cos \varepsilon\sin \beta +\sin \varepsilon\cos \beta \sin \lambda
\end{equation}
\begin{equation}
	\cos \beta \cos \lambda=\cos \alpha \cos \delta
\end{equation}
\begin{equation}
	\cos \delta\sin \alpha=-\sin\beta \sin \varepsilon+\cos \beta \cos \epsilon \sin \lambda
\end{equation}
\subparagraph{由赤道坐标系转化为黄道坐标系}
\begin{equation}
	\sin \beta =\cos \varepsilon \sin \delta -\sin \varepsilon\cos \delta \sin \alpha
\end{equation}
\begin{equation}
	\cos \beta \cos \lambda=\cos \alpha \cos \delta
\end{equation}
\begin{equation}
	\cos \beta \sin \lambda =\sin \delta \sin \varepsilon+\cos \delta \cos \varepsilon \sin \alpha
\end{equation}


\paragraph{银道坐标系和赤道坐标系之间的转化}
设置天体的银道坐标$(l,b)$,赤道坐标$(\alpha,\delta)$。银道面和赤道面夹角$i=62.6^\circ$。银道和赤道升交点的赤经为$\Omega =282^\circ25^\prime$,升交点银经为$l_{\Omega}=33^\circ22.7^\prime$
\subparagraph{由赤道坐标系转化为银道坐标系}
\begin{equation}
	\sin b=\sin \delta \cos i-\cos\delta \sin i \sin (\alpha-\Omega)
\end{equation}
\begin{equation}
	\cos (l-l_{b})\cos b=\cos \delta \cos (\alpha-\Omega)
\end{equation}
\begin{equation}
	\sin(l-l_{\Omega})\cos b=\sin \delta \sin i+\cos \delta \cos i \sin (\alpha-\Omega)
\end{equation}
\subparagraph{由银道坐标系转化为赤道坐标系}
\begin{equation}
	\sin \delta =\sin b \cos i+\cos b \sin i\sin (l-l_{\Omega})
\end{equation}
\begin{equation}
	\cos (\alpha-\Omega)\cos \delta =\cos b\cos (l-l_{\Omega})
\end{equation}
\begin{equation}
	-\sin (\alpha-\Omega)\cos \delta=\sin b\sin i-\cos b\cos i\sin (l-l_{\Omega})
\end{equation}
\paragraph{针对太阳的专用简化公式}
太阳总在赤道上,黄纬始终为零,因此有:
\begin{equation}
	\sin \lambda =\frac{\sin \delta }{\sin \varepsilon}=\frac{\sin \alpha \cos \delta }{\cos \varepsilon}
\end{equation}
\begin{equation}
	\cos \lambda =\cos\alpha\cos \delta 
\end{equation}
\begin{equation}
	\tan \delta =\sin \alpha \tan \varepsilon
\end{equation}

\section{天体升、落、上中天计算}
\paragraph{天体的出没}
\subparagraph{天体的}
赤纬为$\delta$的天体,落山的时角:
\begin{equation}
	\cos t_{set} =-\tan \phi \tan \delta 
\end{equation}
天体升起的时角:
\begin{equation}
	t_{rise}=24^h-t_{set}
\end{equation}
落山时的方位角:(此处方位角的度量方向为南点向西起计算。)
\begin{equation}
	\cos A = -\frac{\sin \delta }{\cos \phi}
\end{equation}
如果已知地平高度$h=90^\circ -z$,那么有:
\begin{equation}
	\sin t =\frac{\sin A\cos h}{\cos \delta }
\end{equation}
\begin{equation}
	\sin \delta=\sin\phi \sin h-\cos \phi \cos h\cos A 
\end{equation}

\paragraph{天体的上中天}
天体在上中天时,时角为0,下中天时为$12^h$。

天体上中天的地平高度:
\begin{equation}
	h=90^\circ -\lvert \phi -\delta \rvert
\end{equation}

天体下中天的地平高度:
\begin{equation}
	h=\lvert\phi -\delta \rvert -90^\circ
\end{equation}
\noindent 若$\delta<\phi$时,天体在天顶以南上中天,同时有:$h=90^\circ-\delta +\phi$

\noindent $\delta >\phi$时,天体在天顶以北上中天,同时有:$h=90^\circ +\delta -\phi$

永不上升天体赤纬的取值范围:$\delta \leq \phi -90^\circ$。

永不下落天体的赤纬取值范围:$\delta > 90^\circ -\phi$。

当恒星上升时,周日视运动圈和地平圈构成$\theta$角,若$\phi$为当地纬度,$\delta $为恒星赤纬,则:
\begin{equation}
	\cos \theta =\sin \phi \sec \delta 
\end{equation}

\chapter{历法与时间计量系统}
\section{恒星时与平太阳时}
\subsection{恒星时}
恒星时是用春分点的周日视运动来定义的。对于某一地点的子午圈,当春分点刚好上中天,即通过子午圈时,即定义为该地地方恒星时零时。

容易从赤经和时角之间的关系看出,某一恒星正在上中天的时候,时角为零,赤经和当时当地的恒星时相等。

\subsection{平太阳时}
\paragraph{太阳时}
太阳刚好通过南方子午圈的时刻定义为太阳时$12^h$。对于任意时刻,将太阳的时角用时分秒为单位度量后加上$12^h$后得到当时的太阳时。太阳时与恒星时间隔因太阳的周年视运动而不同,平均太阳时24小时比恒星时24小时长$3^m56^s$。
\paragraph{真太阳时}
用真实太阳的时角计量的时间称作真太阳时。
\paragraph{平太阳时}
设定如下假想的天体为平太阳:沿着赤道做周期为一个回归年的均匀周年视运动。
以满足此条件的天体——平太阳时角所定义的时间称作平太阳时。平太阳时和真太阳时之差称作时差,一年中时差四次为零,四次达到极值。

\begin{center}
	\begin{tabularx}{\textwidth}{|c|X|X|X|X|X|X|X|X|}
		%\caption{时差表}
		\hline
		日期 &2月12日&4月16日&5月15日&6月15日&7月26日&9月1日&11月3日&12月26日\\
		\hline 
		时差 &$-14^m24^s$&0&$+3^m48^s$&0&$-6^m18^s$&0&$+16^m24^s$&0\\
		\hline 
	\end{tabularx}
\end{center}
平太阳时是假想的,不能直接进行观测。
\paragraph{恒星时和平时的换算}
假设在格林尼治地方,春分点和平太阳都是赤道上的两个点。根据恒星时的定义有:
\begin{equation}
	s=R_{u}+t_{avsun}
\end{equation}

其中,平太阳的赤经为$R_{u}$,$u_{T}$为世界时,即格林尼治地方的平太阳时。平太阳的时角为$t_{avsun}$。平太阳的赤经由以下公式严格定义:
\begin{equation}
	R_{u}=A+PT_{u}+QT_{u}^2+RT_{u}^3
\end{equation}
其中,

	$$A=18^h41^m50.50841^s$$
	$$P=8640184^s.812866$$
	$$Q=0^s.093104$$
	$$R=-6^s.2\times 10^{-6}$$
	$$T_{u}=\frac{JD-2451545.0}{36525}$$
每年世界各地的平时和恒星时完全重合的日期在秋分前后,有且仅有一次。

在精度不高时,可以认为$R_{u}$的表达式中每年元旦中午12时$T_{u}$为零。元旦子夜的恒星时应为$18^h12^m$之前的$12^h2^m$,即$6^h40^m$。从元旦起的第$d$天,平时零时的恒星时$s_{0}$满足:
\begin{equation}
	s_{0}=6^h40^m+3^m.94d
\end{equation}
平时为零时的恒星时表:
\begin{center}

\begin{tabularx}{\textwidth}{|c|X|X|X|X|X|X|}
	%\caption{平时为零的恒星时表:表甲}
	\hline
	日期&1月&2月&3月&4月&5月&6月\\
	\hline
	1&$6^h40^m$&$8^h42^m$&$10^h33^m$&$12^h35^m$&$14^h33^m$&$16^h36^m$\\
	\hline
	4&$6^h52^m$&$8^h54^m$&$10^h45^m$&$12^h47^m$&$14^h35^m$&$16^h47^m$\\
	\hline
	7&$7^h04^m$&$9^h06^m$&$10^h56^m$&$12^h59^m$&$14^h57^m$&$16^h59^m$\\
	\hline
	10&$7^h16^m$&$9^h18^m$&$11^h08^m$&$13^h11^m$&$15^h09^m$&$17^h11^m$\\
	\hline
	13&$7^h28^m$&$9^h30^m$&$11^h20^m$&$13^h22^m$&$15^h21^m$&$17^h23^m$\\
	\hline
	16&$7^h39^m$&$9^h42^m$&$11^h32^m$&$13^h34^m$&$15^h32^m$&$17^h35^m$\\
	\hline
	19&$7^h51^m$&$9^h53^m$&$11^h44^m$&$13^h46^m$&$15^h44^m$&$17^h47^m$\\
	\hline
	22&$8^h03^m$&$10^h05^m$&$11^h56^m$&$13^h58^m$&$15^h56^m$&$17^h58^m$\\
	\hline
	25&$8^h15^m$&$10^h17^m$&$12^h07^m$&$14^h10^m$&$16^h08^m$&$18^h10^m$\\
	\hline
	28&$8^h27^m$&$10^h29^m$&$12^h19^m$&$14^h21^m$&$16^h20^m$&$18^h22^m$\\
	\hline
\end{tabularx}
\end{center}

\begin{center}
\begin{tabularx}{\textwidth}{|c|X|X|X|X|X|X|}
	%\caption{平时为零的恒星时表:表乙}
	\hline
	日期&7月&8月&9月&10月&11月&12月\\
	\hline
	1&$18^h34^m$&$20^h36^m$&$22^h38^m$&$0^h37^m$&$2^h39^m$&$4^h37^m$\\
	\hline
	4&$18^h46^m$&$20^h48^m$&$22^h50^m$&$0^h48^m$&$2^h51^m$&$4^h49^m$\\
	\hline
	7&$18^h57^m$&$21^h00^m$&$23^h02^m$&$1^h00^m$&$3^h02^m$&$5^h01^m$\\
	\hline
	10&$19^h09^m$&$21^h12^m$&$23^h14^m$&$1^h12^m$&$3^h14^m$&$5^h13^m$\\
	\hline
	13&$19^h21^m$&$21^h23^m$&$23^h26^m$&$1^h24^m$&$3^h26^m$&$5^h24^m$\\
	\hline
	16&$19^h33^m$&$21^h35^m$&$23^h37^m$&$1^h36^m$&$3^h38^m$&$5^h36^m$\\
	\hline
	19&$19^h45^m$&$21^h47^m$&$23^h49^m$&$1^h47^m$&$5^h50^m$&$5^h48^m$\\
	\hline
	22&$19^h57^m$&$21^h59^m$&$0^h01^m$&$1^h59^m$&$4^h02^m$&$6^h00^m$\\
	\hline
	25&$20^h08^m$&$22^h11^m$&$0^h13^m$&$2^h11^m$&$4^h13^m$&$6^h12^m$\\
	\hline
	28&$20^h20^m$&$22^h22^m$&$0^h25^m$&$2^h23^m$&$4^h25^m$&$6^h23^m$\\
	\hline
\end{tabularx}
\end{center}
\section{区时与世界时}
\paragraph{区时和世界时}
本部分请读者自行回忆高中地理中自然地理部分内容。
可能的知识点包括但不限于:区时的概念、国际日期变更线、格林尼治时间。
\paragraph{协调世界时}
\subparagraph{协调世界时}一种以原于时秒长为基础,在时刻上尽量接近于世界时的一种折衷的时间系统。协调世界时的秒长严格等于原子时的秒长,采用闰秒的办法使协调时与世界时的时刻相接近。协调世界时的秒小数是国际原子时TAI,秒以上的时、分是世界时UT。当TAI和UT的秒小数之差接近1秒时,将UTC的整数秒增加或减少1秒,此过程分别称作正闰秒和负闰秒。跳秒只有可能发生在某年协调世界时6月30日或12月31日。

本世纪进行的所有闰秒,均为正闰秒(UTC增加1秒)。

\subparagraph{原子时秒的定义}海平面上铯-133原子基态的两个超精细能级间在零磁场下跃迁辐射

9192631770周所持续的时间。

\section{历法}
\subsection{现行公历}
现行公历,又称格里高利历,以耶稣诞生之年作为纪年的开始。它的前身是儒略历。

儒略历是由罗马共和国独裁官儒略·凯撒(又译盖乌斯·尤里乌斯·凯撒)采纳埃及亚历山大的数学家兼天文学家索西琴尼的计算后,于公元前45年1月1日起执行的取代旧罗马历法的一种历法。

儒略历中,一年被划分为12个月,大小月交替;四年一闰,平年365日,闰年366日为在当年二月底增加一闰日,年平均长度为365.25日。由于实际使用过程中累积的误差随着时间越来越大,1582年教皇格里高利十三世颁布、推行了以儒略历为基础改善而来的格里历,即公历。

置闰规则:除非能被400整除,所有的世纪年(能被100整除)都不设闰日;此外所有能被4整除的年份设置闰日。
\subsection{农历}
我国农历是辛亥革命之前实行的传统历法,已经有几千年的历史了。年月和节气的安排完全以月相盈亏和太阳周年视运动两个自然周期为依据,没有人为干预。

农历历法规定:以月相朔所在的那一天为每个月的初一。朔望周期不是日的整倍数,有时可能会出现连续几个大月(30天)或小月(29)天的情况。

我国农历的年以回归年为依据,通过置朔的方法调节年、月两个自然周期,并以二十四节气补充调和日月两个天体运动的自然节律,是一种体系完备的阴阳合历。

置闰的规定根据节气确定。二十四节气是太阳周年视运动过程中黄经每成为$15^\circ$倍数的时刻。从冬至开始,每间隔一个节气称作中气,共十二个:冬至、大寒、雨水、春分、谷雨、小满、夏至、大暑、处暑、秋分、霜降、小雪。春分时,太阳的黄经为零。此后没经过一个节气,太阳的黄经增加$15^\circ$。如果某个朔望月中不包含中气,则这个月份算作上一个月的闰月。但是在地球经过近日点附近,朔望月长度常常大于中气间隔,可能出现一个月中没有中气的情况,故补充规定:如果一个朔望月中有两个中气,则该月份之后的一个或两个月中,没有中气的月份不是闰月。

至于哪个中气所在的月份为岁首,各个朝代有所不同。十二个中气,从冬至开始分别和地支相配,称作月建,正月为岁首。夏朝、商朝、周朝、秦朝分别以建寅、建丑、建子、建亥之月为岁首。汉武帝太初元年颁布太初历,规定岁首依夏历,设在建寅之月(即雨水节气所在的月份)。以后除王莽、曹魏明帝曹叡、武则天和唐肃宗李亨曾短时间有过改动外,一直延续到现在。

由于节气以角度均分,时间间隔不均匀,闰月多发生在四至八月,二月、三月、九月、十月少有发生,出现闰十一月和闰正月极其罕见。如上一次闰十二月出现在明万历二年(1572),上一次闰正月出现在明崇祯十三年(1640),上一次闰十一月出现在,明崇祯十五年(1642)。

\subsection{儒略日}
儒略日是从世界时公元前4713年1月1日中午12点起算的日数。

\subsection{儒略日和公历日期的转换算法}
设公历日期$Y$年$M$月$D$日$H$时。本算法适用于1901$\sim$2099年。

若$M>2$,令$y=Y$,$m=M-3$。
否则,$y=Y-1$,$m=M+9$。
$\left[x\right]$表示对$x$取整数部分。则:

\begin{equation}
	JD=1721103.5+\left[365.25y\right]+\left[30.6m+0.5\right]+D+\frac{H}{24}
\end{equation}
\subsection{给定公历日期,干支纪年、月、日的推算}
\subparagraph{干支纪年}
干支纪年从东汉章帝元和二年(公元85年乙酉)四分历开始。对天干和地支分别给与序号:

对于任意公元年数:天干序号=公元年尾数。地支序号=公元年数除12的余数。

对于任意公元前年数:天干序号=-(公元前年尾数)+11,若大于等于10则减去10。

地支序号=-(公元前年数+1)除12的余数+12,若等于12则为零。
\begin{center}
	\begin{tabularx}{\textwidth}{|c|X|X|X|X|X|X|X|X|X|X|X|X|}
		%\caption{天干、地支序号表}
		\hline
		干支&0&1&2&3&4&5&6&7&8&9&10&11\\
		\hline
		天干&庚&辛&壬&癸&甲&乙&丙&丁&戊&己&&\\
		\hline
		地支&申&酉&戌&亥&子&丑&寅&卯&辰&巳&午&未\\
		\hline
	\end{tabularx}	
\end{center}
\subparagraph{干支纪月}
干支纪月相对简单,各月的的地支是固定不变的,月的天干和年的天干有关。具体内容见下表。
\begin{center}
	\begin{tabularx}{\textwidth}{|c|X|X|X|X|X|X|}
		%\caption{月份的天干、地支和年的天干之间的对应}
		\hline
		月份&月的天干:戊癸&甲乙&乙庚&丙辛&丁壬&月地支\\
		\hline
		1&甲&丙&戊&庚&壬&寅\\
		\hline
		2&乙&丁&己&辛&癸&卯\\
		\hline
		3&丙&戊&庚&壬&甲&辰\\
		\hline
		4&丁&己&辛&癸&乙&巳\\
		\hline
		5&戊&庚&壬&甲&丙&午\\
		\hline
		6&己&辛&癸&乙&丁&未\\
		\hline
		7&庚&壬&甲&丙&戊&申\\
		\hline
		8&辛&癸&乙&丁&己&酉\\
		\hline
		9&壬&甲&丙&戊&庚&戌\\
		\hline
		10&癸&乙&丁&己&辛&亥\\
		\hline
		11&甲&丙&戊&庚&壬&子\\
		\hline
		12&乙&丁&己&辛&癸&丑\\
		\hline
	\end{tabularx}
\end{center}
\subsection{干支纪日}
本方法来自《天文学新概论》第四版。

定义1583年1月1日的"日序"为零。根据古代日食资料,1582年12月25日(明万历十年,壬午年)曾发生日全食,那一天的干支为乙酉。由此看出,1583年1月1日,日的干支为壬辰。

定义此后任意公元年元旦的日序$X_{year}$。注:$\left[x\right]$表示对$x$取整数部分。
\begin{equation}
	X_{year}=365\times D+\left[\frac{D+2}{4}\right]-P
\end{equation}
其中$D$是公元年与1583年的差。P为400年少置三次闰年的改正量,在1600年3月1日以后才予以考虑,此前P值一律为零。
\begin{equation}
	P=(yy-16)-\left[\frac{yy-16}{4}\right]
\end{equation}
$yy$为公元年除100得到的整数。注意:任意一年的一月和二月应该按照前一年对待。

一年中以每月1日为零起点增加的日序:
\begin{equation}
	X_{month}=30\times(m-1)+Q
\end{equation}
其中,$m$为月份,$Q$为不同月份的增加值,按照平年和闰年查下表得出。
\begin{center}
	\begin{tabularx}{\textwidth}{|c|X|X|X|X|X|X|X|}
		\hline
		平年月份&3&1,4,5&2,6,7&8&9,10&11,12&-\\
		\hline
		闰年月份&-&1,3&2,4,5&6,7&8&9,10&11,12\\
		\hline
		$Q$值&-1&0&1&2&3&4&5\\	
		\hline
	\end{tabularx}
\end{center}
每月$d$日以零为起点的日序$X_{day}$:
\begin{equation}
	X_{day}=d-1
\end{equation}
公元某年的日序为:
\begin{equation}
	X_{ymd}=X_{year}+X_{month}+X_{day}
\end{equation}
根据日序计算日的干支:
日的天干:日序尾数+2,若大于等于10则减去10。

日的地支=日序除12的余数+8,若大于等于12则减去12。
\subsection{干支纪时}
\begin{center}
	\begin{tabularx}{\textwidth}{|c|X|X|X|X|X|X|}
		%\caption{时的天干}
		\hline
		小时&日的天干:甲己&乙庚&丙辛&丁壬&戊癸&时地支\\
		\hline
		23$\sim$1&甲&丙&戊&庚&壬&子\\
		\hline
		1$\sim$3&乙&丁&己&辛&癸&丑\\
		\hline
		3$\sim$5&丙&戊&庚&壬&甲&寅\\
		\hline
		5$\sim$7&丁&己&辛&癸&乙&卯\\
		\hline
		7$\sim$9&戊&庚&壬&甲&丙&辰\\
		\hline
		9$\sim$11&己&辛&癸&乙&丁&巳\\
		\hline
		11$\sim$13&庚&壬&甲&丙&戊&午\\
		\hline
		13$\sim$15&辛&癸&乙&丁&己&未\\
		\hline
		15$\sim$17&壬&甲&丙&戊&庚&申\\
		\hline
		17$\sim$19&癸&乙&丁&己&辛&酉\\
		\hline
		19$\sim$21&甲&丙&戊&庚&壬&戌\\
		\hline
		21$\sim$23&乙&丁&己&辛&癸&亥\\
		\hline
	\end{tabularx}
\end{center}
\section{我国古代历法沿革概述}
\paragraph{上古时代}
古人日出而作,日落而息,就是以太阳的升落作为作息时间的客观依据。先人最早认识的时间单位应该是"日"。人们逐渐发现月亮的圆缺周期约为30日。所谓观象授时,就是从物候——草木枯荣、动物迁徙和出入等现象的观察入手,随后过渡到某些星象的观测。

传说在颛顼帝时代,已设立"火正"专司大火星(心宿二)的贯穿 ,以黄昏时分大火星正好从东方地平线上升起时,作为一年的开始。这大约是公元前2400年发生的事,是观象授时的初始形态。

根据《尚书·尧典》记载,在传说中的尧帝时期,乃命羲、和兄弟分别观测鸟、火、虚、昴四颗恒星在黄昏时正上中天的日子,以确定春分、夏至、秋分和冬至,借此划分一年四季。这是我国古代应用阴阳历的初始历法的最早记载,大约发生在公元前2000年。
\paragraph{三代时期}
通过甲骨文的卜辞可知,殷商时期的历法是初具规模的阴阳历。此时,年已有平年、闰年之分,分别为12个月和13个月。一年的长度大约已经用圭表测量确定。新月被定为一个月的开始,月有大月和小月,分别为30日和29日。偶尔有连续大月的情况出现,说明此时人们已经得知朔望月长度应略大于29.5日。这时的岁首已经基本固定,季节和月名都有了基本固定的对应关系。但此时闰月依靠经常性的观测修订,尚未脱离观象授时形态影响。

西周仍然使用阴阳历。周天子通过每年向诸侯国"颁朔"以向诸侯国颁布第二年朔闰的安排及相应的政令。这同时说明,西周时期已将朔日作为一个月的开始。
\paragraph{春秋战国时期}
春秋末期,孔丘在杞国夏人故地得访《夏小正》,这是一个集物候历、观象授时法和初始历法于一,包含物候、天象、气象、农事的作品。并不能将其认定为夏朝行用的历日制度。

大约在战国时期兴起的月令,是《夏小正》历法系统的直接继承者,是阴阳家的立法主张与治国方略的结合体。

到春秋战国之交,一种取回归年长度为$365\frac{1}{4}$日,并采用19年7闰为闰周的古四分历产生。由此可知朔望月的长度为$29\frac{449}{940}$日。
\paragraph{秦汉时期}
从战国到西汉早期,古四分历的历法系统不断完善,吸收阴阳家、占星家等的研究成果,将关于二十四节气、十二个月太阳所在宿次和昏旦中星,以及关于交食和五星位置的初始推算作为历法研究的内容,奠定了我国古代历法的基础。

西汉武帝太初元年(公元前104年),邓平、落下闳等人制定了太初历,(时司马迁为太史令,事可见《史记》卷第一百三十《太史公自序》第七十)。该历法经由西汉末刘歆改造为三统历(汉成帝绥和二年,公元前7年),是为我国现存首部完整的历法。太初历及三统历明确采用以不包含中气的月份定为闰月的方法;引进了交食周期和交点年长度的概念及具体数据;建立了以上元为历元、并以此作为推算气、朔和五星位置等的共同起算点的具体方法;定出了新的五星会合周期及五星在一个会合周期之内的动态表,以及在此基础上预测五星位置的方法;引用了二十四节气太阳所宿度表和二十八宿赤道宿表等。我国古代历法的基本形式更加明确,走向不容逆转。

东汉四分历由编䜣、李梵等人编成,于东汉章帝元和二年(85年)颁行。纠正了太初历(三统历)回归年和朔望历长度均偏大的弊病,使其恢复至古四分历的水平;纠正了冬至点位于牵牛初度的错误,给出了新的测值。东汉四分历同时给出了二十四节气黄道去极度(即太阳视赤纬)、晷影长度、昼夜漏刻长度和昏旦中星的天文表格、二十八宿黄道宿度表格,及其相应的计算任意时日这些天文量的方法,从而充实了我国古代历法的内涵。

东汉末刘洪的乾象历(206,东汉献帝建安十一年),又充实了一批新概念和新方法,如近距历元法的采用,月亮运动不均匀性改正述职表(月离表)和白道离黄道内外数值表(月行阴阳表,即月亮极黄纬表),黄赤道度差表的引进,回归年、朔望月、恒星月、交点食周期和近点月长度的测定,具体计算任意时刻月亮距离黄白交点的度距和太阳所在位置的方法的建立(这实际上已经解决了交食食分大小和交食亏起方位等的计算问题)。至此,我国古代历法的气、朔、闰、晷、漏、交食、五星和恒星位置等广泛论题的计算已经齐全,而且我国古代的历法体系已经成熟。
\paragraph{三国两晋南北朝时期}
我国古代历法在魏晋南北朝时期得到进一步的充实。这表现在:

曹魏杨伟景初历(237年,曹魏明帝景初元年)关于交食食分大小和亏起方位计算方法的明确提出以及多历元的采用,该方法被刘宋何承天元嘉历(443,北魏太武帝太平真君四年,刘宋文帝元嘉二十年)、北魏张龙祥正光历(518,北魏孝明帝熙平三年、神龟元年;南朝梁天监十七年)、李业兴和平历(540,西魏大统六年,东魏兴和二年,南朝梁大同六年)发展和继承;后秦姜岌(384,后秦白雀元年,东晋孝武帝太元九年)测算太阳所在宿度的月食冲法的发明;北凉赵匪攵(此字左右结构,读如"匪")所制玄始历(412,北凉永安十二年/玄始元年,东晋安帝义熙八年)对于闰周的改革;刘宋祖冲之将虞喜于公元330年左右发现的岁差现象引入大明历(463,北魏文成帝和平四年,刘宋孝武帝大明七年)之中,建立了与回归年不同的恒星年概念和数值,提高日月五星在恒星之间位置推算的经度。祖冲之还发明了多测点冬至时刻算法,提高了冬至时刻测量精度。上元历元法得到了祖冲之的阐述与肯定;北魏张龙祥正光历最先将七十二物候引入历法中。

北齐张子信在570年(北周武帝天和五年,北齐后主武平元年,南朝陈宣帝太建二年)前后获得的三大发现:太阳、五星运动的不均匀性,及月球视差对于日食的影响,是我国古代历法史上划时代的事件。
\paragraph{隋唐时期}
隋朝刘焯和张胄玄将张子信的发现引入各自的历法——皇极历(604,隋文帝仁寿四年)和大业历(607,隋炀帝大业三年),于是有太阳和五星运动不均匀性改正表的出现,以及与之相应的同时考虑日月运动不均匀性影响的定朔表,和先求五星平见,次推五星常见(加上太阳运动不均匀的改正),再算五星定见(加上五星运动不均匀的改正)方法的产生;有考虑五星运动不均匀的五星动态表的编制确定;有日应食不食和日不应食而食表的编订,交食食分大小的确定法以及从定朔时刻到食甚时刻改正法的提出;月食食限的不偏食限,必偏食限,不全食限。皇极历另外还有交食初亏、复圆时刻的计算,黄白道度差算法等的创新,对五星动态、昼夜漏刻、黄赤道度差和交食限的计算,创用了等差级数法。大业历另有太阳出入时刻表及相当精确的五星会合周期值的测定。

传统的采用平朔法的观念,在傅仁均戊寅历(619,唐高祖武德二年)和李淳风麟德历(665,唐高宗麟德二年)的出现改变为定朔法。一行大衍历(728,唐玄宗开元十六年)首创不等间距二次差内插法,提出五星运动轨道面同黄道面有一定夹角,五星近日点进动的概念与具体数据,建立了一套推九服晷长、昼夜漏刻长度和食差的近似算法。徐昂宣明历(821,唐穆宗长庆元年)首创日食时差、气差和加差算法,规范了月亮视差对日食食分和日食食时的影响。

自郭献之五纪历(762,唐肃宗宝应元年)开始,出现了将历表公式化的最初尝试。符天历(曹士为,第三字为上下结构,草字头,读如"委",约780)在计算太阳运动不均匀性改正值时首创二次函数式。边冈在崇玄历(892,唐昭宗景福元年)另外给出计算黄赤道度差、月亮极黄纬、日食时差改正、交食初亏与复圆时刻等的二次函数算是,对于晷长计算的三次函数式,对于太阳视赤纬和昼夜漏刻长度的推算使用四次函数算式。
\paragraph{五代十国、辽宋夏金元时期}这一阶段我国古代历法发展的最主要特征是精确化,高次函数法得到了更普遍的应用,计算精度提高。

宋行谷崇天历(1024,北宋仁宗天圣二年)首创了黄白和赤白度差的公式算法。周琮明天历(1064,北宋英宗治平元年)是历代历法中公式化程度最高者,均采用多项式算法计算,其中最高次数为计算晷长的五次多项式。
姚舜辅纪元历(1106)所使用的多项式具有简明、高精度的特点。郭守敬、王恂等人的授时历把三次内插法应用于日月五星运动诸课题的计算。

授时历的另外一项重要改革,是采用实测历元法,同时得到有关天文量的各不相同的起算点。这项改革吸取了南宋杨忠辅统天历(1199)的基本模式。
\paragraph{明清时期}
明朝开始,历法出现倒退。明朝沿用授时历,只是改名大统历而已,偶尔有人提出改换历法均被以"祖制不可变"而被否定。这一时期来自阿拉伯的回回历法受到部分天文官员的了解及重视。明代晚期,试图恢复传统历法的努力和传入的西方历法开始了论争与融汇。

明崇祯二年至七年(1629$\sim$1634),由徐光启等人主持,耶稣会士龙华民、汤若望等人参与编撰《崇祯历书》,较为系统地介绍了西方经典天文学的重大成果。满清入关后,汤若望将《崇祯历书》改名为《西洋新法历书》上献,虽然清朝有人试图重振传统历法,但没有对传统历法的颓势做出实质上的改变。