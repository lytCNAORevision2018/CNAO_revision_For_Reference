\chapter{自序}
我还是犹豫了一下,决定我终究还是要在这里写点什么的。根据闷声发大财的基本原则,我应该尽量少表达自己的观点,尤其是在这份花了很大心血的复习资料的前言部分。但是在这里什么都不写,真的,一点也不好。

首先说一下这份复习资料吧。这份复习资料的第一版正式发布于2017年8月31日,那是一个排版出了很大问题的Word文档。我在2019年参加中考,但是被当地最好高中的重点班而不是实验班录取;2022年参加高考,踩着某所江苏省的211院校在黑龙江省的录取分数线开始了自己的大学生活。而在从2009年7月22日——令我虽然只能看到转播的电视画面但是记忆犹新的日全食的那一天,到2022年6月30日——正式在高考的志愿填报中放弃了一切和天文或物理有关的选择,天文,或者和天文有关的活动,在我的空闲时间内占据了很大的比例。当我回忆起这一段时光时,我总会觉得那是我的第二条生命。而现在,情况完全不一样了。它不再是我生命的一半。除了四六级、托福雅思和与本专业紧密相关的编程技术,更多的时间和精力在逐渐给几个考研的科目代号让路:101,201,301,838/877/960。当然,做出这项决定时非常痛苦的,可是也没办法。再次选择的成本很高,高到我不一定能负担得起——不仅是经济上的,还有思想上的。

我还能清晰地记得,当初撰写这份复习资料究竟是为了什么——全国中学生天文知识竞赛。但是当时还不叫这个名字,叫"全国中学生天文知识竞赛",当时国家对各类学科竞赛还没有这样严格的管理措施,有的小学生也能在决赛中取得很好的成绩。2017年3月25日,我前往哈尔滨的预赛考点——哈尔滨工业大学附属中学初中部,参加第一次天文奥赛的预赛。预赛的结果,败北。第二年再次在三月的最后一个周六,还是下午两点钟到四点半,还是败北。而第一次CNAO败北,是我开始修撰这份复习资料的直接原因;而连续两次的败北,是我牵头组建CNAO复习充电站并自任站长的直接原因。

无论如何,虽然我的大学所在的城市的GDP超过了某大学天文系所在的同一省的省会城市,虽然我的专业对物理的要求并不高,似乎我余生都不太可能再和天文专业有什么过多的交集,我还是把这份第十六版的复习资料做了出来。虽然我也很怀念2017年八月盛夏坐在屏幕前和Word公式编辑器厮守的那段时光,为了确保大量的公式能够准确地排版,撰写这个版本复习资料的时候还是使用了\LaTeX。高考后本来我有时间去沉下心来独自修撰这份复习资料,并在同一时间熟悉\LaTeX 的操作。但是在高考交卷后,尤其是填报志愿过后产生的超大空窗期的十字路口,我选择了《史记》和《汉书》,选择了《颜勤礼碑》和Python。

希望阅读本资料的读者已经通读过《天文学新概论》或《基础天文学》,对必要天文知识和常识已经有了必要的了解,同时熟悉天文奥赛的考察大纲。不然,读者有可能在阅读本参考资料的时候寸步难行。有的人会认为这可以是一本教材的雏形,笔者认为,完全没有必要将其视作某本教材的雏形。它更像是一本被涂涂改改很多次的备忘录或者一幅绘制得过于详细的地图,缺乏可读性,加之笔者水平有限,故将其再改写为一本适合有一定数学和物理基础的中学生准备天文奥赛的教材的计划是不可行的。

读者需要注意的是,因为时间仓促和编者水平有限,本资料在以下方向上具有较为严重的缺陷。
$\left(1\right)$缺乏对大爆炸宇宙模型下宇宙演化过程各个阶段的大致描述。请参考《天文学新概论》第四版的378至381页。$\left(2\right)$本资料没有对暗物质和暗能量做必要的介绍。也没有介绍宇宙学发展的过程以及各种宇宙模型。$\left(3\right)$本资料没有对我国古代的天文知识做一个提纲挈领式的梳理。具体请参考王力等所编写《古代汉语》第三册的有关部分。$\left(4\right)$本资料因编撰时间紧张,缺乏图像。

对于以上各缺陷,望各位读者海涵。

我总在想,天文这项爱好,对我来说究竟意味着什么。如果她,是我的一位好朋友,她在我心目中又究竟有个什么样的定位。录取通知书收到后,每每想到天文,我总能想到《哀江南赋》。天文对于我,"班超生而望返,温序死而思归"
,是永不折返的故园的光阴。"连茂苑于海陵,跨横塘于江浦。东门则鞭石成桥,南极则铸铜为柱。橘则园植万株,竹则家封千户。西赆浮玉,南琛没羽。吴歈越吟,荆艳楚舞 。草木之遇阳春,鱼龙之逢风雨" ,我经历过。但现在我的精神状态更倾向于"闻鹤唳而心惊,听胡笳而泪下。拒神亭而亡戟,临横江而弃马。崩于钜鹿之沙,碎于长平之瓦"。我只得"乘白马而不前,策青骡而转碍"再"入欹斜之小径,掩蓬藋之荒扉。就汀洲之杜若,待芦苇之单衣"。有可能这会成为我一生的遗憾。在这种情况下,还和之前修订各版本的复习资料一样地全身心投入、不计时间精力成本地撰写这版复习资料,只能为减轻自己的遗憾做一点微不足道的工作了。

说实话,我并不期待这份资料能在未来的某个时候、某个地方被用到。如果用到,我会非常的开心。但是,如发现错误,还请及时和笔者联系。本着负责到底的态度我会尽全力订正的。
\begin{flushright}
	
开府仪同三司\quad 使持节都督天文奥赛复习充电站诸军事\quad 李轩辕\quad 自撰
\end{flushright}

\begin{flushright}
	2023年2月4日\qquad 立春
\end{flushright}